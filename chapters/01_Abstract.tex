\begin{abstract}

    Diese Seminararbeit untersucht den Einfluss des generativen Designs auf kreative Gestaltungsprozesse. Durch die Analyse der Grundlagen, Methoden, Anwendungen und Herausforderungen des generativen Designs wird die zentrale Fragestellung beantwortet: Wie unterstützt generatives Design die kreativen Gestaltungsprozesse?

    Im Verlauf der Arbeit werden die Grundlagen erläutert, wobei die Definition, ein historischer Überblick, die Rolle von KI in generativem Design und die benötigten Technologien dazu erklärt werden.
    
    Ein zentraler Schwerpunkt liegt auf den Methoden und dem Designprozess des generativen Designs. Unterschiedliche Ansätze und Techniken werden analysiert, um die Umsetzung der Designmethoden in kreative Gestaltungsprozesse zu veranschaulichen. Dabei wird nur auf computergestützte Methoden eingegangen und das unter der Annahme das ein Grundwissen über Künstliche Intelligenz besteht.
    
    Des Weiteren werden Anwendungen des \ac{gD} in verschiedenen Branchen betrachtet, um den Mehrwert zu verdeutlichen. Ein Fallbeispiel aus der Architektur sowie Anwendungen in anderen Bereichen werden präsentiert. Darüber hinaus erfolgt eine Vorstellung des Unternehmens Autodesk sowie seiner Softwarelösungen für Generatives Design, um einen umfassenden Einblick in den aktuellen technischen Stand zu geben.
    
    Es werden die Herausforderungen erörtert und Ethische sowie rechtliche Aspekte diskutiert. Zum Abschluss werden auch die Auswirkungen technologischer Entwicklungen auf das \ac{gD} untersucht.

\end{abstract}