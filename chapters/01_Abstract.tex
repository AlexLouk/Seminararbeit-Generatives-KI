\begin{abstract}


    Diese Seminararbeit untersucht den Einfluss des Generativen Designs auf kreative Gestaltungsprozesse in der Designbranche. Die zentrale Fragestellung lautet: Wie beeinflusst das Generative Design den Entscheidungsprozess und die kreative Intuition der Designer? Um diese Frage zu beantworten, werden verschiedene Methoden des Generativen Designs untersucht, darunter parametrisches Design, algorithmisches Design, evolutionäre Algorithmen, prozedurale Generierung, Simulation und Analyse, Machine Learning und Künstliche Intelligenz, generative Algorithmen sowie datengesteuertes Design.
    
    Im Rahmen dieser Seminararbeit werden die grundlegenden Konzepte und Eigenschaften des Generativen Designs erläutert. Dabei liegt der Fokus auf mathematischen Modellen, Regelsystemen und Algorithmen, die zur Beschreibung und Manipulation von formalen und ästhetischen Eigenschaften von Designs verwendet werden. Es werden Fallbeispiele und Anwendungen des Generativen Designs in verschiedenen Bereichen wie Architektur, Produktgestaltung, Grafikdesign, Kunst, Modedesign, Industriedesign sowie Medizin und Gesundheitswesen untersucht.
    
    Die Ergebnisse dieser Untersuchung zeigen, dass das Generative Design einen signifikanten Einfluss auf die kreativen Gestaltungsprozesse hat. Es ermöglicht Designern, effizienter zu arbeiten, Materialersparnisse zu erzielen und innovative Lösungen zu generieren. Durch die Integration von automatisierten Prozessen und algorithmischen Methoden eröffnet das Generative Design neue Wege der Gestaltung, die über traditionelle Designansätze hinausgehen.
    
    Die Folgerungen aus dieser Seminararbeit sind vielfältig. Das Generative Design bietet Potenziale für eine effizientere Gestaltung, die Reduzierung des Materialverbrauchs, die Förderung von Innovationen und eine hohe Anpassungsfähigkeit. Es ermöglicht es Designern, verschiedene Variationen und Optionen zu erkunden und maßgeschneiderte Lösungen für unterschiedliche Nutzerbedürfnisse zu entwickeln. Die Erkenntnisse dieser Arbeit haben Implikationen für die Designpraxis und bieten Anwendungsmöglichkeiten in verschiedenen Bereichen.
    
    Insgesamt liefert diese Seminararbeit ein umfassendes Verständnis für das Generative Design und seine Auswirkungen auf die Designbranche. Sie bietet eine Grundlage für die Diskussion über die Zukunft der kreativen Gestaltungsprozesse und zeigt Wege auf, wie das Generative Design in der Praxis angewendet werden kann, um innovative Lösungen zu generieren.

\end{abstract}