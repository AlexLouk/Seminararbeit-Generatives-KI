\begin{abstract}

Diese Seminararbeit untersucht das Thema des generativen Designs und seine Auswirkungen auf die kreativen Gestaltungsprozesse in der Designbranche. Der Fokus liegt auf der Definition und den Konzepten des generativen Designs sowie dem historischen Überblick über seine Entwicklung. Darüber hinaus werden verschiedene Methoden des generativen Designs wie parametrisches Design, algorithmisches Design, evolutionäre Algorithmen und datengesteuertes Design vorgestellt. Es werden auch die Anwendungen des generativen Designs in Bereichen wie Architektur, Produktgestaltung, Grafikdesign, Modedesign, Industriedesign, Medizin und Gesundheitswesen untersucht. Die Herausforderungen und Zukunftsaussichten des generativen Designs werden ebenfalls diskutiert, einschließlich ethischer und rechtlicher Aspekte sowie technologischer Entwicklungen. Schließlich werden die Ergebnisse dieser Arbeit zusammengefasst, die Forschungsfrage beantwortet und eine kritische Bewertung sowie ein Ausblick auf die zukünftige Bedeutung des generativen Designs in der Designbranche gegeben. Diese Arbeit trägt dazu bei, das Verständnis und die Wertschätzung des generativen Designs als innovativen Ansatz für kreative Gestaltungsprozesse zu vertiefen.


\end{abstract}