\subsection*{Ethische und rechtliche Aspekte}
Durch das generative Design ergeben sich einige ethische und rechtliche Fragestellungen. Eine bedeutende im Bereich der Ethik betrifft die Zuordnung der Originalität und Urheberschafft von generierten Werken. Wessen Eigentum ist das entstandene Werk, da das Werk auf Algorithmen und computergenerierten Prozessen basiert, stellt sich die Frage, ob der Algorithmus oder der Designer der Schöpfer des Werkes ist. Das wirft Fragen hinsichtlich geistiger Eigentumsrechte auf. 
Die nächste Frage, die sich stellt, welche Auswirkungen hat das generative Design hinsichtlich der Arbeit und des Berufslebens? Durch die Automatisierung der Erstellung von Design-Lösungen fällt einige Arbeitszeit weg oder komplette Arbeitsplätze. Traditionelle kreative Berufsfelder könnten komplett wegfallen. Gesellschaftlich muss überlegt werden, wie man damit umgeht. 
Rechtlich interessant wird es beim Thema Haftung und Verantwortung im Falle von Fehlern oder Schäden im Zusammenhang mit generierten Lösungen. Wer trägt die Verantwortung, wenn ein Algorithmus oder die KI-Software versagt? Hat der Anwender nicht ausreichend geprüft und der Software blind vertraut oder gibt es einen Fehler in der Software? Hier muss ein klarer rechtlicher Rahmen geschaffen werden, um potenzielle Streitigkeiten zu verhindern und dem Anwender klare Vorgaben geben. 
Ein weiterer rechtlicher Kritikpunkt betrifft mögliche Verletzungen des geistigen Eigentums. Es können Werke oder Designs erstellt werden, die Ähnlichkeiten mit urheberrechtlich geschützten Werken aufweisen. Dies kann unbeabsichtigt und beabsichtigt passieren. Es sollte sorgfältig überprüft werden ob Werke gegen bestehende Eigentumsrechte verstoßen. \autocite{6}
\subsection*{Technologische Entwicklung}
Das Generative Design ist noch eine relative Junge Disziplin. Es steht in engem Zusammenhang mit technologischen Weiterentwicklungen. 
Mit der kontinuierliche steigenden Rechenleistung werden komplexe Generierungsalgorithmen und Simulationen schneller und effizienter. Dies ermöglicht die Verarbeitung wesentlich größerer Datenmengen für noch präzisere Ergebnisse.
Die Integration von KI-Technologien wie maschinelles Lernen oder Deep Learning eröffnet spannende Perspektiven. Generative Algorithmen können dazu lernen, Muster zu erkennen, Vorlieben von Menschen zu verstehen und auf Basis dieser Erkenntnisse bessere Lösungen zu generieren. 
Fortschritte in der 3D-Drucktechnologie ermöglicht die Erstellung gestalteter Objekte und Strukturen direkt aus den digitalen Modellen. Das ermöglicht neue Möglichkeiten um komplexe Designlösungen, die mit herkömmlichen Produktionsmöglichkeiten nur sehr aufwändig erstellten werden können, sehr einfach, materialsparend und kostengünstig zu erstellen. 
\ac*{vr} und \ac*{ar}: \ac*{vr}- und \ac*{ar}-Technologien eröffnen die Möglichkeit zur Visualisierung und Interaktion mit generativem Design. Der Anwender kann eine virtuelle Umgebung nutzen, um seine Ideen zu visualisieren und testen, bevor sie physisch umgesetzt werden müssen. \ac*{ar} ermöglicht, generative Lösungen in die reale Welt zu projizieren und in verschiedenen Kontexten zu betrachten. Dies ermöglicht ein komplett anderes Design-Feedback. 
Der Zugriff auf immer größer werdende Datenmengen und der Fortschritt in der damit zusammenhängenden Datenanalyse ermöglicht die Erstellung von individuelleren Lösungen. Durch die Analyse von Benutzerdaten, Trends und anderen relevanten Informationen können  generative Algorithmen personalisierte Modelle erstellen und auf spezifischere Wünsche besser reagieren. \autocite{10}
