Unter dem Oberbegriff des generativen Designs sind verschiedene Methoden zu finden, die sich teilweise stark voneinander unterscheiden. Je nach Branche und Designziel werden unterschiedliche Methoden angewendet. Im Folgenden werden einige aktuelle Designmethoden näher beschrieben.

\subsection*{Parametrisches Design}
Parametrisches Design ist eine Methode, bei der Modelle auf einer Reihe von Parametern basieren. Diese Parameter sind variabel und können Eigenschaften wie Größe, Form, Proportionen, Materialien und andere designrelevante Merkmale eines Objekts oder einer Struktur repräsentieren. Bei der Anwendung des parametrischen Designs werden zunächst die Parameter festgelegt, die den Raum der möglichen Designs definieren. Anschließend werden Algorithmen oder Regeln entwickelt, die diese Parameter beeinflussen und miteinander in Beziehung setzen. Durch die Manipulation dieser Parameter können Designer verschiedene Variationen und Iterationen des Designs erzeugen. Der große Vorteil des parametrischen Designs liegt in seiner Flexibilität und Effizienz. Indem die Designentscheidungen auf Parameter abgebildet werden, können Änderungen an einem Parameter automatisch zu Änderungen im gesamten Design führen. Dies ermöglicht eine schnelle Exploration verschiedener Designoptionen und eine einfache Anpassung an veränderte Anforderungen.Darüber hinaus ermöglicht das parametrische Design auch die Optimierung von Designs. Durch die Verwendung von Optimierungsalgorithmen können Designer bestimmte Ziele oder Kriterien festlegen, die das Design erfüllen soll. Der Algorithmus sucht dann automatisch nach den besten Parametereinstellungen, um diese Ziele zu erreichen.
Parametrisches Design wird vor allem in Branchen eingesetzt, die die Entwicklung komplexer und maßgeschneiderter Designs erfordern und auf spezifische Anforderungen zugeschnitten werden müssen wie in der Architektur oder im Produktdesign.

\subsection*{Evolutionäre Algorithmen}
Diese Designmethode ist aus den Prinzipien der biologischen Evolution inspiriert. Sie ermöglicht automatisierte Generierung und Optimierung von Designs, indem sie eine vorher bestimmte Population an Designs erzeugt und diese iterativ weiterentwickelt.
Denn Start hat die erste Population dargestellt die aus einer zufälligen Auswahl an möglichen Designs, aus einem Zufälligen Satz an Parametern erstellt wurde. Aus dieser Population wird von dem Designer oder von einer \ac*{ki} die kreiirten Designs mit einem \textit{Fitness}-Wert versehen. Dies kann der Designer alternativ auch über eine selbst programmierte Fitnessfunktion machen in der er die Kriterien und Ziele angibt die das Enddesign haben sollen. So wird kein menschlicher Input mehr benötigt bis man zu einem Ergebnis kommt. Aus den Bewerteten Designs wird nun die zweite Generation von Designs erstellt. Diese zweite Generation besitzt die Eigenschaften der Designs aus der ersten Generation, die einen hohen Fitness-Wert hatten. Dieser Prozess wird so oft wiederholt bis ein zufriedenstellendes Ergebnis erzielt wurde. Je höher die Iterationen der Designs, umso angepasster sind die fertigen Designs an den Anforderungen. 

\subsection*{Generative Adversarial Networks (GANs)}

