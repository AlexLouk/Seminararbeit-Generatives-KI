A. Architektur und Bauwesen:
1. Parametrisches Design: Ein Architekt nutzt parametrisches Design, um automatisch verschiedene Variationen eines Gebäudes zu generieren, indem er Parameter wie Größe, Form und Material anpasst. Dadurch kann er schnell verschiedene Entwürfe erstellen und deren Auswirkungen analysieren.

2. Simulation und Analyse: Ein Architekt nutzt Simulationen, um den Energieverbrauch und die thermische Leistung eines Gebäudes zu analysieren und das Design entsprechend anzupassen, um eine optimale Energieeffizienz zu erreichen. Durch die Nutzung von Analysewerkzeugen kann das Design iterativ optimiert werden.

B. Produktgestaltung:
1. Algorithmisches Design: Ein Produktgestalter nutzt algorithmisches Design, um automatisch verschiedene Produktvarianten zu generieren. Durch die Festlegung von Regeln und Variationen in Form, Farbe und Anordnung kann der Designer schnell eine Vielzahl von Designoptionen erkunden und bewerten.

2. Machine Learning und Künstliche Intelligenz: Ein Unternehmen für Produktgestaltung nutzt maschinelles Lernen, um aus vorhandenen Daten zu lernen und neue Designs zu generieren. Es können Muster, Stile oder Präferenzen aus einer Vielzahl von Beispielen gelernt werden, um personalisierte und auf die Bedürfnisse der Benutzer zugeschnittene Produkte zu entwerfen.

C. Grafikdesign und Kunst:
1. Algorithmisches Design: Ein Grafikdesigner nutzt algorithmisches Design, um automatisch verschiedene Logo-Designs zu generieren. Durch die Festlegung von Regeln und Variationen in Form, Farbe und Anordnung können vielfältige Designoptionen erkundet werden.

2. Generative Algorithmen: Ein Künstler verwendet generative Algorithmen, um abstrakte Kunstwerke zu generieren. Durch die Festlegung von Regeln für Formen, Farben und Bewegungen entstehen einzigartige und dynamische Ergebnisse.

D. Modedesign:
1. Prozedurale Generierung: Ein Modedesigner nutzt prozedurale Generierung, um automatisch Muster für Stoffe oder Texturen zu erstellen. Durch die Anwendung von wiederholbaren Verfahren können vielfältige und komplexe Designs erzeugt werden.

2. Machine Learning und Künstliche Intelligenz: Ein Modelabel verwendet maschinelles Lernen, um aus einer großen Menge von Modefotos neue Designs zu generieren. Die künstliche Intelligenz erkennt Muster, Stile und Trends in den Daten und erstellt darauf basierend neue Kleidungsstücke.

E. Industriedesign:
1. Parametrisches Design: Ein Industriedesigner nutzt parametrisches Design, um automatisch verschiedene Variationen eines Produkts zu generieren, indem er Parameter wie Größe, Form und Material anpasst. Dadurch können schnell alternative Designoptionen erforscht werden.

2. Datengesteuertes Design: Ein Industriedesigner verwendet datengesteuertes Design, um die Benutzererfahrung zu verbessern. Durch die Analyse von Benutzerdaten und -präferenzen kann das Design eines Produkts an die Bedürfnisse und Vorlieben der Benutzer angepasst werden.

F. Medizin und Gesundheitswesen:
1. Simulation und Analyse: Ein Medizintechnikunter

nehmen nutzt Simulationen, um die Leistung und Wirksamkeit eines medizinischen Geräts zu analysieren. Dadurch können iterative Verbesserungen am Design vorgenommen werden, um eine optimale Leistung und Sicherheit zu gewährleisten.

2. Machine Learning und Künstliche Intelligenz: Ein Unternehmen für medizinische Geräteentwicklung nutzt maschinelles Lernen, um aus einer großen Menge von Patientendaten Designs für personalisierte medizinische Geräte zu generieren. Die individuellen Bedürfnisse und Präferenzen der Benutzer werden dabei berücksichtigt.

