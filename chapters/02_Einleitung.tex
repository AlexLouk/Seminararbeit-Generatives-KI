Generatives Design mit künstlicher Intelligenz hat in den letzten Jahren einen großen Einfluss auf das Produktdesign und 
die Fertigung von Produkten gewonnen. Es ermöglicht Unternehmen, innovative und maßgeschneiderte Lösungen für ihre Kunden 
zu schaffen, indem es automatisiert verschiedene Designoptionen generiert und optimiert. Dabei wird ein Satz von Parametern 
und Kriterien definiert, die das Design beeinflussen, und dann werden unzählige Design-Optionen von der KI generiert, die die 
Vorgaben erfüllen. Anschließend können die besten Optionen ausgewählt werden, um das endgültige Produkt zu entwickeln.
Dieser Prozess bietet eine effiziente Möglichkeit, die Effektivität und Leistung von Produkten zu verbessern und gleichzeitig 
den Materialverbrauch und die Herstellungskosten zu reduzieren. Es hat sich gezeigt, dass Unternehmen, die generatives Design 
und künstliche Intelligenz nutzen, ihre Produkte schneller auf den Markt bringen können, wettbewerbsfähiger sind und bessere 
Kundenzufriedenheit erreichen.

Ein herausragendes Beispiel für den Einsatz von generativem Design mit künstlicher Intelligenz ist der Nike Flyprint-Schuh. 
Nike hat in Zusammenarbeit mit Autodesk das Design-Tool entwickelt, um den Schuh durch generatives Design zu entwerfen. Der 
Schuh wurde speziell für Athleten entwickelt und sollte eine optimale Passform und Leistung bieten. Durch die Verwendung von 
generativem Design mit künstlicher Intelligenz konnte Nike schnell und effizient tausende von Design-Optionen generieren und 
die besten Optionen für den Schuh auswählen. Das Ergebnis war ein innovativer Schuh, der den Anforderungen von Athleten gerecht
 wurde und gleichzeitig den Materialverbrauch reduzierte. 

Nike's Flyprint-Schuh, der durch generatives Design mit künstlicher Intelligenz entworfen wurde, ist nur ein Beispiel für die 
vielen Anwendungen von künstlicher Intelligenz im Produktdesign. Doch wie genau wird generatives Design mit künstlicher Intelligenz
eingesetzt und welche Auswirkungen hat es auf die Produktdesign-Branche?