\subsection*{Problemstellung}
Generatives Design mit künstlicher Intelligenz hat in den letzten Jahren einen großen Einfluss auf das Produktdesign und 
die Fertigung von Produkten gewonnen. Es ermöglicht Unternehmen, innovative und maßgeschneiderte Lösungen für ihre Kunden 
zu schaffen, indem es automatisiert verschiedene Designoptionen generiert und optimiert. Dabei wird ein Satz von Parametern 
und Kriterien definiert, die das Design beeinflussen, und dann werden unzählige Design-Optionen von der KI generiert, die die 
Vorgaben erfüllen. Anschließend können die besten Optionen ausgewählt werden, um das endgültige Produkt zu entwickeln.
Dieser Prozess bietet eine effiziente Möglichkeit, die Effektivität und Leistung von Produkten zu verbessern und gleichzeitig 
den Materialverbrauch und die Herstellungskosten zu reduzieren. Es hat sich gezeigt, dass Unternehmen, die generatives Design 
und künstliche Intelligenz nutzen, ihre Produkte schneller auf den Markt bringen können, wettbewerbsfähiger sind und bessere 
Kundenzufriedenheit erreichen.

Ein herausragendes Beispiel für den Einsatz von generativem Design mit künstlicher Intelligenz ist der Nike Flyprint-Schuh. 
Nike hat in Zusammenarbeit mit Autodesk das Design-Tool entwickelt, um den Schuh durch generatives Design zu entwerfen. Der 
Schuh wurde speziell für Athleten entwickelt und sollte eine optimale Passform und Leistung bieten. Durch die Verwendung von 
generativem Design mit künstlicher Intelligenz konnte Nike schnell und effizient tausende von Design-Optionen generieren und 
die besten Optionen für den Schuh auswählen. Das Ergebnis war ein innovativer Schuh, der den Anforderungen von Athleten gerecht
 wurde und gleichzeitig den Materialverbrauch reduzierte. 

Nike's Flyprint-Schuh, der durch generatives Design mit künstlicher Intelligenz entworfen wurde, ist nur ein Beispiel für die 
vielen Anwendungen von künstlicher Intelligenz im Produktdesign. Doch wie genau wird generatives Design mit künstlicher Intelligenz
eingesetzt und welche Auswirkungen hat es auf die Produktdesign-Branche?

\subsection*{Zielsetzung}
Die vorliegende Seminararbeit hat zum Ziel, den Einfluss des generativen Designs auf kreative Gestaltungsprozesse in der Designbranche zu untersuchen. Dabei sollen die grundlegenden Konzepte und Methoden des generativen Designs erläutert werden, um ein umfassendes Verständnis für diese innovative Designpraxis zu vermitteln. Zudem sollen konkrete Anwendungen des generativen Designs in verschiedenen Bereichen wie Architektur, Produktgestaltung, Grafikdesign, Kunst, Modedesign, Industriedesign sowie Medizin und Gesundheitswesen untersucht werden. 

Die Arbeit befasst sich ebenfalls mit den Herausforderungen, denen das generative Design gegenübersteht, und bietet einen Ausblick auf zukünftige Entwicklungen und potenzielle Innovationen. Dabei werden ethische und rechtliche Aspekte im Zusammenhang mit generativem Design berücksichtigt. 

Durch eine umfassende Literaturrecherche und Analyse soll die Forschungsfrage beantwortet werden: "Wie beeinflusst generatives Design die kreativen Gestaltungsprozesse in der Designbranche?" Dabei werden die Auswirkungen von generativem Design auf die Kreativität und den Gestaltungsprozess untersucht und kritisch bewertet. 

Die Ergebnisse dieser Arbeit sollen dazu beitragen, ein besseres Verständnis für die Möglichkeiten und Herausforderungen des generativen Designs in der Designbranche zu gewinnen und einen Beitrag zur Diskussion über die Zukunft der kreativen Gestaltungsprozesse zu leisten.

\subsection*{Aufbau der Arbeit}
Der Aufbau der Arbeit folgt einer logischen Struktur, die es dem Leser ermöglicht, die Entwicklung des Themas nachzuvollziehen. Nach einer einführenden Einleitung werden in Kapitel II die Grundlagen des generativen Designs erläutert, um ein solides Fundament für das weitere Verständnis zu schaffen. Kapitel III widmet sich den verschiedenen Methoden des generativen Designs und gibt einen Überblick über ihre Funktionsweise.

Kapitel IV beschäftigt sich mit den konkreten Anwendungen des generativen Designs in verschiedenen Bereichen, wobei für jeden Bereich Fallbeispiele präsentiert werden, um die praktische Anwendung zu veranschaulichen. In Kapitel V werden die Herausforderungen und Zukunftsaussichten des generativen Designs diskutiert, wobei ethische, rechtliche und technologische Aspekte betrachtet werden.

Abschließend erfolgt in Kapitel VI eine Zusammenfassung der Ergebnisse, die Beantwortung der Forschungsfrage sowie eine kritische Bewertung und ein Ausblick auf zukünftige Entwicklungen. Die Arbeit wird mit einem Literaturverzeichnis abgeschlossen, das die verwendeten Quellen angibt.




