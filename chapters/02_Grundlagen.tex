\subsection*{Definition}
Das Generative Design ist ein innovativer Ansatz, bei dem Algorithmen und computergestützte Methoden in den Gestaltungsprozess integriert werden. Es ermöglicht Designern, mithilfe vordefinierter Regeln und Parametern automatisch Variationen und Iterationen von Designs zu generieren (Siehe \autoref{chap:paramDesign}). Im Zentrum steht die Idee, den Computer als kreativen Partner einzubeziehen, um komplexe und innovative Lösungen zu entwickeln, die über traditionelle manuelle oder konventionelle Ansätze hinausgehen.

Eine wichtige Methode im Generativen Design ist die Anwendung parametrischer Modelle. Diese Modelle beschreiben mathematische Zusammenhänge und Regelsysteme, die sowohl die formale als auch ästhetische Eigenschaften von Designs beschreiben und manipulieren können. Durch den Einsatz von Algorithmen und automatisierten Prozessen können Designer effizienter arbeiten und schnell verschiedene Variationen und Optionen erkunden, um neue Perspektiven zu gewinnen und innovative Lösungen zu entwickeln.

\subsection*{Materialersparnisse und Ressourcenoptimierung im Generativen Design}
Ein bedeutendes Ziel des Generativen Designs liegt in den potenziellen Materialersparnissen und der Ressourcenoptimierung. Durch die Integration algorithmischer Methoden und parametrischer Modelle kann das Generative Design dazu beitragen, effizientere und ressourcenschonendere Designs zu entwickeln.

Durch den Einsatz generativer Designwerkzeuge können Designer komplexe Strukturen und Formen optimieren, um Materialverschwendung zu minimieren. Das Generative Design berücksichtigt Belastungen, Spannungen und andere physikalische Anforderungen und gestaltet Designs so, dass sie die benötigte Festigkeit und Stabilität aufweisen, während unnötiges Material entfernt wird. Dadurch können erhebliche Materialersparnisse erzielt werden.

Darüber hinaus eröffnet das Generative Design Möglichkeiten für die Entwicklung von Leichtbaustrukturen, bei denen Material nur dort platziert wird, wo es benötigt wird. Dies führt zu einer erheblichen Reduzierung des Materialverbrauchs und kann zu Gewichtseinsparungen führen, was insbesondere in Bereichen wie der Luft- und Raumfahrt, der Automobilindustrie und der Architektur von großer Bedeutung ist.

Ein weiterer Aspekt ist die Optimierung der Materialwahl. Durch die Fähigkeit des Generativen Designs, komplexe Optimierungen und Simulationen durchzuführen, können Designer alternative Materialien und Materialkombinationen untersuchen, um die Effizienz und Nachhaltigkeit der Designs weiter zu verbessern. Dies ermöglicht es, umweltfreundlichere Materialien einzusetzen und den Einsatz von Ressourcen zu optimieren.

Die Integration von Generativem Design in den Gestaltungsprozess kann somit erhebliche Vorteile hinsichtlich Materialersparnis und Ressourcenoptimierung bieten, was zu nachhaltigeren und effizienteren Designlösungen führt. \autocite*{20}

\subsection*{Historischer Überblick}
Der historische Überblick des Generativen Designs reicht bis in die 1960er und 1970er Jahre zurück, als erste Experimente mit computergestützter Gestaltung durchgeführt wurden. Zu dieser Zeit begannen Designer und Forscher, den Einsatz von Algorithmen und computergestützten Methoden zu erkunden, um kreative Prozesse zu unterstützen.

In den folgenden Jahrzehnten wurden erhebliche Fortschritte in der Computertechnologie und der Algorithmik erzielt, was zu einer breiteren Anwendung generativer Designmethoden führte. Insbesondere mit dem Aufkommen leistungsfähiger Computer und der Entwicklung spezialisierter Designsoftware wurde das Potenzial des Generativen Designs weiter ausgeschöpft.

Heutzutage ist generatives Design in verschiedenen Bereichen der Gestaltung verbreitet. Dabei werden spezifische generative Designmethoden verwendet, um die jeweiligen Anforderungen und Herausforderungen in den einzelnen Bereichen zu bewältigen. \autocite*{18}

\subsection*{Rolle von \ac{ki} in Generativen Design}
Künstliche Intelligenz (\ac{ki}) spielt eine entscheidende Rolle im generativen Design, da sie die Fähigkeiten von Designern erweitert und den kreativen Prozess unterstützt. Durch den Einsatz von \ac{ki}-Technologien wie maschinellem Lernen und Deep Learning können Designalgorithmen große Mengen an Daten analysieren, Muster erkennen und neue Designlösungen generieren. \ac{ki} ermöglicht es, komplexe Zusammenhänge und Anforderungen zu berücksichtigen und gleichzeitig innovative und effiziente Designs zu schaffen.

Ein wichtiger Aspekt ist die Optimierung von Designs. \ac{ki}-basierte Algorithmen können die Topologieoptimierung unterstützen (Siehe \autoref{chap:topology}), um Materialien und Strukturen zu identifizieren, die die gewünschten Leistungsmerkmale erfüllen. Durch die Simulation und Bewertung verschiedener Designoptionen kann \ac{ki} helfen, optimale Lösungen zu finden, die herkömmlichem Design möglicherweise entgehen würden.

Darüber hinaus ermöglicht \ac{ki} auch die Integration von Benutzerpräferenzen und Designvorgaben. Durch das Lernen aus Nutzerfeedback und historischen Daten können \ac{ki}-Systeme personalisierte Designempfehlungen geben und den Designprozess auf die individuellen Bedürfnisse und Vorlieben der Benutzer abstimmen.

Die Rolle von \ac{ki} im generativen Design geht jedoch über die Automatisierung und Unterstützung von Designaufgaben hinaus. Sie eröffnet auch neue Möglichkeiten für kreative Exploration und die Schaffung neuartiger Designs. \ac{ki} kann dabei helfen, Designräume zu erforschen, unkonventionelle Lösungen zu identifizieren und innovative Konzepte zu generieren, die traditionelle Designansätze herausfordern.

Insgesamt trägt \ac{ki} dazu bei, den Designprozess effizienter, vielfältiger und kreativer zu gestalten. Sie unterstützt Designer dabei, neue Ideen zu generieren, Designräume zu erkunden und optimale Lösungen zu finden, die den Anforderungen und Präferenzen der Benutzer gerecht werden. Durch die enge Verbindung von \ac{ki} und generativem Design eröffnen sich spannende Perspektiven für die Gestaltung zukünftiger Produkte und Systeme. \autocite*{21} \autocite*{22}

\subsection*{Benötigte Technologien für Generatives Design}
Generatives Design erfordert oft große Rechenleistung und Speicherplatz, insbesondere bei komplexen Projekten. Aus diesem Grund ist Cloud-Computing eine geeignete Lösung, da es skalierbare Ressourcen in Form von virtuellen Maschinen und Speicherplatz bietet. Durch die Nutzung von Cloud-Computing können Designer auf leistungsstarke Recheninfrastruktur zugreifen und große Datenmengen effizient verarbeiten.

Ein weiterer wichtiger Aspekt ist das High-Performance Computing (HPC). Da generatives Design rechenintensiv ist und viele Iterationen und Optimierungsschritte erfordert, können HPC-Systeme mit Mehrkernprozessoren und paralleler Verarbeitung die Rechenzeiten erheblich verkürzen. Diese Systeme können entweder in einer Cloud-Infrastruktur oder lokal betrieben werden, je nach den Anforderungen des Projekts.

<<<<<<< HEAD
Generatives Design benötigt eine Menge an Daten, wie zum Beispiel topografische Informationen (siehe Kapitel xxx), Gebäudeparameter und Materialdaten. Daher ist eine robuste Infrastruktur für das Datenmanagement und die Integration von entscheidender Bedeutung. Dies umfasst die Einrichtung von Datenbanken, die Entwicklung von Datenpipelines und die Integration von Daten aus verschiedenen Quellen, um effektiv mit Daten umgehen zu können. Da große Unternehmen oft an verschiedenen Standorten verteilt sind, ist eine gute Netzwer\ac{ki}nfrastruktur notwendig. 
=======
Generatives Design benötigt eine Menge an Daten, wie zum Beispiel topografische Informationen \autoref{chap:topology}, Gebäudeparameter und Materialdaten. Daher ist eine robuste Infrastruktur für das Datenmanagement und die Integration von entscheidender Bedeutung. Dies umfasst die Einrichtung von Datenbanken, die Entwicklung von Datenpipelines und die Integration von Daten aus verschiedenen Quellen, um effektiv mit Daten umgehen zu können. Da große Unternehmen oft an verschiedenen Standorten verteilt sind, ist eine gute Netzwerkinfrastruktur notwendig. 
>>>>>>> 034787a76bf3cc1b490b48b0944f2ad8d5afc0bc

Ein weiterer wichtiger Aspekt ist die Versionierung und Zusammenarbeit. Bei generativem Design ist es entscheidend, den Überblick über die verschiedenen Designiterationen zu behalten und eine nahtlose Zusammenarbeit zwischen den Teammitgliedern zu ermöglichen. Hier kommen Versionierungssysteme und kollaborative Plattformen zum Einsatz, die die Verwaltung und Zusammenarbeit an gemeinsamen Projekten ermöglichen. Diese bieten eine Historie und erleichtern die Koordination zwischen Teams.

