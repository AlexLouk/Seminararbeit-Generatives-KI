
1. Parametrisches Design: Die Verwendung von parametrischen Modellen, bei denen Designelemente und -parameter miteinander verknüpft sind. Durch die Anpassung dieser Parameter können verschiedene Designvarianten generiert werden. Beispiel: Ein Architekt nutzt parametrisches Design, um automatisch verschiedene Variationen eines Gebäudes zu generieren, indem er Parameter wie Größe, Form und Material anpasst.

2. Algorithmisches Design: Die Anwendung von Algorithmen zur Generierung von Designs. Diese Algorithmen können Regeln, Bedingungen und Zufallselemente enthalten, um unterschiedliche Ergebnisse zu erzielen. Beispiel: Ein Grafikdesigner nutzt algorithmisches Design, um automatisch verschiedene Logo-Designs zu generieren, indem er Regeln und Variationen in Form, Farbe und Anordnung festlegt.

3. Evolutionäre Algorithmen: Die Anwendung von genetischen oder evolutionären Algorithmen, um Designs zu generieren und zu optimieren. Dabei werden Designvarianten erzeugt, bewertet und miteinander kombiniert, um immer bessere Ergebnisse zu erzielen. Beispiel: Ein Fahrzeughersteller verwendet evolutionäre Algorithmen, um verschiedene Fahrzeugdesigns zu generieren und sie basierend auf Kriterien wie Aerodynamik, Effizienz und Ästhetik zu optimieren.

4. Prozedurale Generierung: Die Nutzung von Regeln, Algorithmen oder Programmcode, um automatisch Designs zu erzeugen. Prozedurale Generierung ermöglicht die Erzeugung von komplexen und vielfältigen Designs, indem wiederholbare Verfahren angewendet werden. Beispiel: In der Videospielentwicklung wird prozedurale Generierung verwendet, um automatisch Landschaften, Levels und Charaktere zu erstellen, wodurch eine große Vielfalt an Spielinhalten generiert werden kann.

5. Simulation und Analyse: Die Verwendung von Simulationen und Analysewerkzeugen, um das Verhalten, die Leistung oder andere Aspekte des Designs zu bewerten. Dies ermöglicht eine iterative Optimierung und Verbesserung des Designs. Beispiel: Ein Architekt nutzt Simulationen, um den Energieverbrauch und die thermische Leistung eines Gebäudes zu analysieren und das Design entsprechend anzupassen, um eine optimale Energieeffizienz zu erreichen.

6. Machine Learning und Künstliche Intelligenz: Der Einsatz von maschinellen Lernverfahren und künstlicher Intelligenz, um aus vorhandenen Daten zu lernen und neue Designs zu generieren. Dabei können Muster, Stile oder Präferenzen aus einer Vielzahl von Beispielen erlernt werden. Beispiel: Ein Unternehmen für medizinische Geräteentwicklung nutzt maschinelles Lernen, um aus einer großen Menge von Patientendaten Designs für personalisierte medizinische Geräte zu generieren, die den individuellen Bedürfnissen und Präferenzen der Benutzer entsprechen.

7. Generative Algorithmen: Die Nutzung von spezifischen Algorithmen, die auf generativen Prinzipien basieren, um neue Designs zu erzeugen. Diese Algorithmen können auf Regeln, Wahrscheinlichkeiten oder emergentem Verhalten basieren. Beispiel: Ein Künstler verwendet generative Algorithmen, um abstrakte Kunstwerke zu generieren, indem er Regeln für Formen, Farben und Bewegungen festlegt, die zu einzigartigen und dynamischen Ergebnissen führen.

8. Datengesteuertes Design: Die Verwendung von Daten, um Designs zu generieren oder zu beeinflussen. Dies können beispielsweise Umgebungsdaten, Benutzerpräferenzen oder andere Informationen sein, die in den Generierungsprozess einfließen. Beispiel: Ein Webdesigner nutzt datengesteuertes Design, um die Benutzererfahrung zu verbessern, indem er das Design einer Website basierend auf dem Verhalten der Benutzer anpasst, um deren Bedürfnisse und Vorlieben besser zu erfüllen.