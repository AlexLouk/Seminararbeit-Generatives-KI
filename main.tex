\documentclass[conference]{IEEEtran}

%% DOCUMENT FORMATTING
\usepackage[ngerman]{babel}
\usepackage{csquotes}
\usepackage{geometry}

%% Hyperlinks
\usepackage{hyperref}

%% GRAPHICS
\usepackage{graphicx}

% CITATION
\usepackage{acronym}

\usepackage[style=ieee, maxcitenames=2, mincitenames=1]{biblatex}
\addbibresource{sources.bib}


\def\BibTeX{{\rm B\kern-.05em{\sc i\kern-.025em b}\kern-.08em
    T\kern-.1667em\lower.7ex\hbox{E}\kern-.125emX}}
\begin{document}
\pagenumbering{Roman} 

\title{Generatives KI-Design\\
\large \ \\ \large Eine Untersuchung der Grenzen und Möglickeiten}

\author{
  \IEEEauthorblockN{Alexandros Loukaridis}
  \IEEEauthorblockA{\textit{MatNr. 1000730} \\
  92loal1bif@hft-stuttgart.de}

  \and

  \IEEEauthorblockN{Valentin Franco}
  \IEEEauthorblockA{\textit{MatNr. 380094} \\
  91frva1bif@hft-stuttgart.de}
}

\maketitle

\begin{abstract}

    Diese Seminararbeit untersucht den Einfluss des Generativen Designs auf kreative Gestaltungsprozesse in der Designbranche. Die zentrale Fragestellung lautet: Wie beeinflusst das Generative Design den Entscheidungsprozess und die kreative Intuition der Designer? Um diese Frage zu beantworten, werden verschiedene Methoden des Generativen Designs untersucht, darunter parametrisches Design, algorithmisches Design, evolutionäre Algorithmen, prozedurale Generierung, Simulation und Analyse, Machine Learning und Künstliche Intelligenz, generative Algorithmen sowie datengesteuertes Design.
    
    Im Rahmen dieser Seminararbeit werden die grundlegenden Konzepte und Eigenschaften des Generativen Designs erläutert. Dabei liegt der Fokus auf mathematischen Modellen, Regelsystemen und Algorithmen, die zur Beschreibung und Manipulation von formalen und ästhetischen Eigenschaften von Designs verwendet werden. Es werden Fallbeispiele und Anwendungen des Generativen Designs in verschiedenen Bereichen wie Architektur, Produktgestaltung, Grafikdesign, Kunst, Modedesign, Industriedesign sowie Medizin und Gesundheitswesen untersucht.
    
    Die Ergebnisse dieser Untersuchung zeigen, dass das Generative Design einen signifikanten Einfluss auf die kreativen Gestaltungsprozesse hat. Es ermöglicht Designern, effizienter zu arbeiten, Materialersparnisse zu erzielen und innovative Lösungen zu generieren. Durch die Integration von automatisierten Prozessen und algorithmischen Methoden eröffnet das Generative Design neue Wege der Gestaltung, die über traditionelle Designansätze hinausgehen.
    
    Die Folgerungen aus dieser Seminararbeit sind vielfältig. Das Generative Design bietet Potenziale für eine effizientere Gestaltung, die Reduzierung des Materialverbrauchs, die Förderung von Innovationen und eine hohe Anpassungsfähigkeit. Es ermöglicht es Designern, verschiedene Variationen und Optionen zu erkunden und maßgeschneiderte Lösungen für unterschiedliche Nutzerbedürfnisse zu entwickeln. Die Erkenntnisse dieser Arbeit haben Implikationen für die Designpraxis und bieten Anwendungsmöglichkeiten in verschiedenen Bereichen.
    
    Insgesamt liefert diese Seminararbeit ein umfassendes Verständnis für das Generative Design und seine Auswirkungen auf die Designbranche. Sie bietet eine Grundlage für die Diskussion über die Zukunft der kreativen Gestaltungsprozesse und zeigt Wege auf, wie das Generative Design in der Praxis angewendet werden kann, um innovative Lösungen zu generieren.

\end{abstract}

\section{Einleitung}
Generatives Design mit künstlicher Intelligenz hat in den letzten Jahren einen großen Einfluss auf das Produktdesign und 
die Fertigung von Produkten gewonnen. Es ermöglicht Unternehmen, innovative und maßgeschneiderte Lösungen für ihre Kunden 
zu schaffen, indem es automatisiert verschiedene Designoptionen generiert und optimiert. Dabei wird ein Satz von Parametern 
und Kriterien definiert, die das Design beeinflussen, und dann werden unzählige Design-Optionen von der KI generiert, die die 
Vorgaben erfüllen. Anschließend können die besten Optionen ausgewählt werden, um das endgültige Produkt zu entwickeln.
Dieser Prozess bietet eine effiziente Möglichkeit, die Effektivität und Leistung von Produkten zu verbessern und gleichzeitig 
den Materialverbrauch und die Herstellungskosten zu reduzieren. Es hat sich gezeigt, dass Unternehmen, die generatives Design 
und künstliche Intelligenz nutzen, ihre Produkte schneller auf den Markt bringen können, wettbewerbsfähiger sind und bessere 
Kundenzufriedenheit erreichen.

Ein herausragendes Beispiel für den Einsatz von generativem Design mit künstlicher Intelligenz ist der Nike Flyprint-Schuh. 
Nike hat in Zusammenarbeit mit Autodesk das Design-Tool entwickelt, um den Schuh durch generatives Design zu entwerfen. Der 
Schuh wurde speziell für Athleten entwickelt und sollte eine optimale Passform und Leistung bieten. Durch die Verwendung von 
generativem Design mit künstlicher Intelligenz konnte Nike schnell und effizient tausende von Design-Optionen generieren und 
die besten Optionen für den Schuh auswählen. Das Ergebnis war ein innovativer Schuh, der den Anforderungen von Athleten gerecht
 wurde und gleichzeitig den Materialverbrauch reduzierte. 

Nike's Flyprint-Schuh, der durch generatives Design mit künstlicher Intelligenz entworfen wurde, ist nur ein Beispiel für die 
vielen Anwendungen von künstlicher Intelligenz im Produktdesign. Doch wie genau wird generatives Design mit künstlicher Intelligenz
eingesetzt und welche Auswirkungen hat es auf die Produktdesign-Branche?

\section{Einführung in generatives Design}
Generatives Design hat in den letzten Jahren immer mehr an Bedeutung gewonnen und verspricht neue Möglichkeiten für die Kreativbranche. Diese innovative Technologie kombiniert künstliche Intelligenz und fortschrittliche Algorithmen, um automatisch kreative Inhalte, Muster und Formen zu generieren, die sowohl ästhetisch ansprechend als auch funktional sind. Dabei wird das Generative Design sowohl im Bereich des Designs als auch in der Konstruktion eingesetzt.

Im Designprozess ermöglicht das Generative Design Designern und Künstlern die Erzeugung einer Vielzahl von Variationen und neuen Ideen. Mithilfe von maschinellem Lernen werden komplexe Muster und Zusammenhänge erkannt, um maßgeschneiderte Designs zu generieren, die den spezifischen Anforderungen gerecht werden. Dies ermöglicht eine effiziente und schnelle Erzeugung von individuellen Designs, die den Bedürfnissen und Anforderungen der Nutzer entsprechen.

In der Konstruktion spielt das Generative Design eine entscheidende Rolle bei der Erstellung optimierter 3D-Modelle. Durch die Integration von Cloud-Computing und künstlicher Intelligenz werden verschiedene Designparameter berücksichtigt, wie beispielsweise Fertigungsprozesse, Belastungen und Einschränkungen. Auf Grundlage dieser Anforderungen bietet die Software passende Designs an. Das Generative Design ermöglicht Ingenieuren die Maximierung der Leistungsfähigkeit eines Produkts unter Berücksichtigung von Gewichtsbeschränkungen, physikalischen Einschränkungen und der Verfügbarkeit von Materialien.

Generatives Design bietet somit eine innovative Möglichkeit, optimierte 3D-Modelle mithilfe von künstlicher Intelligenz zu erstellen. Es erleichtert Designern und Ingenieuren die Arbeit, spart Zeit und eröffnet neue Gestaltungsmöglichkeiten. Durch die Verbindung von künstlicher Intelligenz, kreativem Denken und technischer Innovation kann das Generative Design einen positiven Einfluss auf die Design- und Konstruktionsbranche haben.

\subsection*{Definition}
Die Definition von generativem Design bezieht sich auf eine Technologie oder einen Ansatz, bei dem Algorithmen und künstliche Intelligenz verwendet werden, um automatisch kreative Lösungen oder Designs zu generieren. Dabei werden bestimmte Parameter und Anforderungen festgelegt, auf deren Grundlage die Software oder der Algorithmus eine Vielzahl von möglichen Designs oder Lösungen erstellt. Generatives Design nutzt das Potenzial des maschinellen Lernens, um aus großen Datenmengen zu lernen und optimierte Ergebnisse zu erzeugen, die den gestellten Anforderungen entsprechen. Es ermöglicht eine effiziente und schnelle Erzeugung von Designs, die den individuellen Bedürfnissen und Anforderungen gerecht werden.

\subsection*{Methoden und Anwendungsgebiete}
1. Parametrisches Design: Die Verwendung von parametrischen Modellen, bei denen Designelemente und -parameter miteinander verknüpft sind. Durch die Anpassung dieser Parameter können verschiedene Designvarianten generiert werden. Beispiel: Ein Architekt nutzt parametrisches Design, um automatisch verschiedene Variationen eines Gebäudes zu generieren, indem er Parameter wie Größe, Form und Material anpasst.

2. Algorithmisches Design: Die Anwendung von Algorithmen zur Generierung von Designs. Diese Algorithmen können Regeln, Bedingungen und Zufallselemente enthalten, um unterschiedliche Ergebnisse zu erzielen. Beispiel: Ein Grafikdesigner nutzt algorithmisches Design, um automatisch verschiedene Logo-Designs zu generieren, indem er Regeln und Variationen in Form, Farbe und Anordnung festlegt.

3. Evolutionäre Algorithmen: Die Anwendung von genetischen oder evolutionären Algorithmen, um Designs zu generieren und zu optimieren. Dabei werden Designvarianten erzeugt, bewertet und miteinander kombiniert, um immer bessere Ergebnisse zu erzielen. Beispiel: Ein Fahrzeughersteller verwendet evolutionäre Algorithmen, um verschiedene Fahrzeugdesigns zu generieren und sie basierend auf Kriterien wie Aerodynamik, Effizienz und Ästhetik zu optimieren.

4. Prozedurale Generierung: Die Nutzung von Regeln, Algorithmen oder Programmcode, um automatisch Designs zu erzeugen. Prozedurale Generierung ermöglicht die Erzeugung von komplexen und vielfältigen Designs, indem wiederholbare Verfahren angewendet werden. Beispiel: In der Videospielentwicklung wird prozedurale Generierung verwendet, um automatisch Landschaften, Levels und Charaktere zu erstellen, wodurch eine große Vielfalt an Spielinhalten generiert werden kann.

5. Simulation und Analyse: Die Verwendung von Simulationen und Analysewerkzeugen, um das Verhalten, die Leistung oder andere Aspekte des Designs zu bewerten. Dies ermöglicht eine iterative Optimierung und Verbesserung des Designs. Beispiel: Ein Architekt nutzt Simulationen, um den Energieverbrauch und die thermische Leistung eines Gebäudes zu analysieren und das Design entsprechend anzupassen, um eine optimale Energieeffizienz zu erreichen.

6. Machine Learning und Künstliche Intelligenz: Der Einsatz von maschinellen Lernverfahren und künstlicher Intelligenz, um aus vorhandenen Daten zu lernen und neue Designs zu generieren. Dabei können Muster, Stile oder Präferenzen aus einer Vielzahl von Beispielen erlernt werden. Beispiel: Ein Unternehmen für medizinische Geräteentwicklung nutzt maschinelles Lernen, um aus einer großen Menge von Patientendaten Designs für personalisierte medizinische Geräte zu generieren, die den individuellen Bedürfnissen und Präferenzen der Benutzer entsprechen.

7. Generative Algorithmen: Die Nutzung von spezifischen Algorithmen, die auf generativen Prinzipien basieren, um neue Designs zu erzeugen. Diese Algorithmen können auf Regeln, Wahrscheinlichkeiten oder emergentem Verhalten basieren. Beispiel: Ein Künstler verwendet generative Algorithmen, um abstrakte Kunstwerke zu generieren, indem er Regeln für Formen, Farben und Bewegungen festlegt, die zu einzigartigen und dynamischen Ergebnissen führen.

8. Datengesteuertes Design: Die Verwendung von Daten, um Designs zu generieren oder zu beeinflussen. Dies können beispielsweise Umgebungsdaten, Benutzerpräferenzen oder andere Informationen sein, die in den Generierungsprozess einfließen. Beispiel: Ein Webdesigner nutzt datengesteuertes Design, um die Benutzererfahrung zu verbessern, indem er das Design einer Website basierend auf dem Verhalten der Benutzer anpasst, um deren Bedürfnisse und Vorlieben besser zu erfüllen.

\section{Künstliche Intelligenz im generativen Design}
\input{./chapters/04_KIimGD.tex}

\section{Kritische Bewertung}
\subsection*{Positive Aspekte}
Steigerung der Produktivität durch Automatisierung:
KI-gestützte Programme können Routineaufgaben im Designprozess übernehmen, 
was Zeit und Aufwand für Designer einspart. Beispielsweise können Designsysteme, 
Komponenten und Styleguides automatisch generiert werden, was eine schnellere 
Anpassung für verschiedene Prozesse ermöglicht. Dies erlaubt Designern, 
sich auf kreativere Aufgaben zu konzentrieren und effizienter zu arbeiten.

Personalisierung von Inhalten:
KI ermöglicht eine individuelle Anpassung von Inhalten und bietet ein enormes 
Potenzial für eine verbesserte User Experience. Algorithmen analysieren Nutzerverhalten,
um personalisierte Inhalte bereitzustellen. Im Bereich der Design-Psychologie kann KI 
verschiedene Variationen basierend auf bekannten Mustern erstellen und somit 
individuelle Designs generieren. Dies ermöglicht Unternehmen, maßgeschneiderte Erlebnisse 
für ihre Kunden zu schaffen und ihren Markenerfolg zu steigern.

KI-basiertes Testing und Analysen:
KI ermöglicht schnelle und effiziente A/B-Tests, um Produkte bereits im frühen 
Designprozess zu evaluieren und Feedback von einer Vielzahl von Nutzern 
zu erhalten. Funktionen wie die Erkennung von Mimiken, Gesten und Eye-Tracking 
Heatmaps ermöglichen eine objektive Bewertung der Benutzerfreundlichkeit und ermöglichen 
Optimierungen des Designs. Durch KI-gestützte Tests und Analysen können Designer fundierte 
Entscheidungen treffen und das Design kontinuierlich verbessern.

A/B-Tests: Eine experimentelle Methode zur Vergleich von Varianten. Ziel ist es, herauszufinden, welche Variante die besten Ergebnisse erzielt. Dabei werden zwei oder mehr Gruppen gebildet, die unterschiedliche Varianten eines Elements erhalten. Durch Messung von Leistungskennzahlen wie Klickrate oder Conversion-Rate kann die effektivste Variante ermittelt werden. A/B-Tests ermöglichen datenbasierte Designentscheidungen und kontinuierliche Optimierung.

Herausforderungen:
Fehlende Differenzierung von Nuancen:
Eine der Herausforderungen im Bereich des generativen Designs mit KI besteht darin, dass 
die KI Schwierigkeiten hat, Nuancen von menschlichen Emotionen richtig zu differenzieren 
und zu verstehen. Emotionen wie Freude, Trauer, Angst oder Überraschung sind komplexe menschliche 
Erfahrungen, die nicht einfach von einer KI nachgeahmt werden können. Die KI-Modelle können zwar 
Muster erkennen und gewisse Reaktionen vorhersagen, aber sie haben Schwierigkeiten, subtile emotionale 
Unterschiede zu erfassen und angemessen darauf zu reagieren. Die genaue Analyse und Interpretation von 
Gefühlen bleibt eine der größten Schwachstellen der KI im Bereich des Designs.

Bewegenden Content schaffen:
Eine weitere Herausforderung besteht darin, mit generativen Designs emotionale Tiefe und bewegende Inhalte
 zu schaffen. Während KI-Programme in der Lage sind, eine Vielzahl von auf den Nutzer abgestimmten Designvariationen 
 zu generieren, fehlt ihnen oft die Fähigkeit, eine tiefere emotionale Resonanz zu erzeugen. Bei der Generierung von 
 Inhalten werden häufig abstrakte Elemente verwendet, die zwar ästhetisch ansprechend wirken können, aber nicht unbedingt 
 die gleiche emotionale Wirkung und die mitreißende Kraft von beispielsweise handgemalten Gemälden oder individuell 
 gestalteten Werken haben. Das Schaffen von Inhalten mit einer starken emotionalen Verbindung bleibt eine Herausforderung 
 für das generative Design mit KI.

Die Gefahr von Vorurteilen:
Ein weiteres kritisches Thema im Zusammenhang mit KI-gestütztem Design ist die potenzielle Verstärkung von 
Vorurteilen und Diskriminierung. KI-Systeme lernen aus Daten und Mustern, die ihnen zur Verfügung gestellt werden. 
Wenn diese Daten Vorurteile oder unfaire Unterscheidungen enthalten, können sich diese in den generierten Designs 
widerspiegeln. Beispielsweise könnten Entscheidungen aufgrund von Hautfarbe oder Gesichtszügen getroffen werden, 
die zu einer Benachteiligung bestimmter Gruppen führen. Dies wird als "Algorithmic Bias" bezeichnet und stellt eine 
ernsthafte ethische Herausforderung dar. Um solche Vorurteile zu vermeiden, ist es wichtig, von Anfang an ethische 
Prinzipien zu setzen und die Auswahl der Daten sowie die Trainingsmethoden der KI-Modelle sorgfältig zu überwachen.

\subsection*{Negative Aspekte}

\section{Zukunftsperspektiven}
\input{./chapters/06_ZukunftUndFazit.tex}

\listoffigures
\addcontentsline{toc}{section}{Abbildungsverzeichniss}

\section*{Abkürzungsverzeichnis}
\begin{acronym}
  \acro{ki}[KI]{Künstliche Intelligenz}
  \acro{gd}[GD]{Generatives Design }
  \acro{gD}[gD]{generativen Designs}
  \acro{GAN}[GAN]{Generative Adversarial Network}
  \acro{GAN}[GANs]{Generative Adversarial Networks}
  \acro{CAD}[CAD]{computer aided design}

\end{acronym}

\section*{Literaturverzeichnis}
\printbibliography[heading=none]{}

\end{document}
