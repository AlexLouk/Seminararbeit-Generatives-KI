\documentclass[conference]{IEEEtran}

%% DOCUMENT FORMATTING
\usepackage[ngerman]{babel}
\usepackage{csquotes}
\usepackage{geometry}

%% Hyperlinks
\usepackage{hyperref}

%% GRAPHICS
\usepackage{graphicx}

% CITATION
\usepackage{acronym}

\usepackage[style=ieee, maxcitenames=2, mincitenames=1]{biblatex}
\addbibresource{sources.bib}


\def\BibTeX{{\rm B\kern-.05em{\sc i\kern-.025em b}\kern-.08em
    T\kern-.1667em\lower.7ex\hbox{E}\kern-.125emX}}
\begin{document}
\pagenumbering{Roman} 

\title{Generatives KI-Design\\
\large \ \\ \large Eine Untersuchung der Grenzen und Möglickeiten}

\author{
  \IEEEauthorblockN{Alexandros Loukaridis}
  \IEEEauthorblockA{\textit{MatNr. 1000730} \\
  92loal1bif@hft-stuttgart.de}

  \and

  \IEEEauthorblockN{Valentin Franco}
  \IEEEauthorblockA{\textit{MatNr. 380094} \\
  01frva1bif@hft-stuttgart.de}
}

\maketitle

\begin{abstract}

    Diese Seminararbeit untersucht den Einfluss des Generativen Designs auf kreative Gestaltungsprozesse in der Designbranche. Die zentrale Fragestellung lautet: Wie beeinflusst das Generative Design den Entscheidungsprozess und die kreative Intuition der Designer? Um diese Frage zu beantworten, werden verschiedene Methoden des Generativen Designs untersucht, darunter parametrisches Design, algorithmisches Design, evolutionäre Algorithmen, prozedurale Generierung, Simulation und Analyse, Machine Learning und Künstliche Intelligenz, generative Algorithmen sowie datengesteuertes Design.
    
    Im Rahmen dieser Seminararbeit werden die grundlegenden Konzepte und Eigenschaften des Generativen Designs erläutert. Dabei liegt der Fokus auf mathematischen Modellen, Regelsystemen und Algorithmen, die zur Beschreibung und Manipulation von formalen und ästhetischen Eigenschaften von Designs verwendet werden. Es werden Fallbeispiele und Anwendungen des Generativen Designs in verschiedenen Bereichen wie Architektur, Produktgestaltung, Grafikdesign, Kunst, Modedesign, Industriedesign sowie Medizin und Gesundheitswesen untersucht.
    
    Die Ergebnisse dieser Untersuchung zeigen, dass das Generative Design einen signifikanten Einfluss auf die kreativen Gestaltungsprozesse hat. Es ermöglicht Designern, effizienter zu arbeiten, Materialersparnisse zu erzielen und innovative Lösungen zu generieren. Durch die Integration von automatisierten Prozessen und algorithmischen Methoden eröffnet das Generative Design neue Wege der Gestaltung, die über traditionelle Designansätze hinausgehen.
    
    Die Folgerungen aus dieser Seminararbeit sind vielfältig. Das Generative Design bietet Potenziale für eine effizientere Gestaltung, die Reduzierung des Materialverbrauchs, die Förderung von Innovationen und eine hohe Anpassungsfähigkeit. Es ermöglicht es Designern, verschiedene Variationen und Optionen zu erkunden und maßgeschneiderte Lösungen für unterschiedliche Nutzerbedürfnisse zu entwickeln. Die Erkenntnisse dieser Arbeit haben Implikationen für die Designpraxis und bieten Anwendungsmöglichkeiten in verschiedenen Bereichen.
    
    Insgesamt liefert diese Seminararbeit ein umfassendes Verständnis für das Generative Design und seine Auswirkungen auf die Designbranche. Sie bietet eine Grundlage für die Diskussion über die Zukunft der kreativen Gestaltungsprozesse und zeigt Wege auf, wie das Generative Design in der Praxis angewendet werden kann, um innovative Lösungen zu generieren.

\end{abstract}

\section{Einleitung}
Generatives Design mit künstlicher Intelligenz hat in den letzten Jahren einen großen Einfluss auf das Produktdesign und 
die Fertigung von Produkten gewonnen. Es ermöglicht Unternehmen, innovative und maßgeschneiderte Lösungen für ihre Kunden 
zu schaffen, indem es automatisiert verschiedene Designoptionen generiert und optimiert. Dabei wird ein Satz von Parametern 
und Kriterien definiert, die das Design beeinflussen, und dann werden unzählige Design-Optionen von der KI generiert, die die 
Vorgaben erfüllen. Anschließend können die besten Optionen ausgewählt werden, um das endgültige Produkt zu entwickeln.
Dieser Prozess bietet eine effiziente Möglichkeit, die Effektivität und Leistung von Produkten zu verbessern und gleichzeitig 
den Materialverbrauch und die Herstellungskosten zu reduzieren. Es hat sich gezeigt, dass Unternehmen, die generatives Design 
und künstliche Intelligenz nutzen, ihre Produkte schneller auf den Markt bringen können, wettbewerbsfähiger sind und bessere 
Kundenzufriedenheit erreichen.

Ein herausragendes Beispiel für den Einsatz von generativem Design mit künstlicher Intelligenz ist der Nike Flyprint-Schuh. 
Nike hat in Zusammenarbeit mit Autodesk das Design-Tool entwickelt, um den Schuh durch generatives Design zu entwerfen. Der 
Schuh wurde speziell für Athleten entwickelt und sollte eine optimale Passform und Leistung bieten. Durch die Verwendung von 
generativem Design mit künstlicher Intelligenz konnte Nike schnell und effizient tausende von Design-Optionen generieren und 
die besten Optionen für den Schuh auswählen. Das Ergebnis war ein innovativer Schuh, der den Anforderungen von Athleten gerecht
 wurde und gleichzeitig den Materialverbrauch reduzierte. 

Nike's Flyprint-Schuh, der durch generatives Design mit künstlicher Intelligenz entworfen wurde, ist nur ein Beispiel für die 
vielen Anwendungen von künstlicher Intelligenz im Produktdesign. Doch wie genau wird generatives Design mit künstlicher Intelligenz
eingesetzt und welche Auswirkungen hat es auf die Produktdesign-Branche?

\listoffigures
\addcontentsline{toc}{section}{Abbildungsverzeichniss}

\section*{Abkürzungsverzeichnis}
\begin{acronym}
  \acro{ki}[KI]{Künstliche Intelligenz}
  \acro{gd}[GD]{Generatives Design }
  \acro{gD}[gD]{generativen Designs}
  \acro{GAN}[GAN]{Generative Adversarial Network}
  \acro{GAN}[GANs]{Generative Adversarial Networks}
  \acro{CAD}[CAD]{computer aided design}

\end{acronym}

\section*{Literaturverzeichnis}
\printbibliography[heading=none]{}

\end{document}
